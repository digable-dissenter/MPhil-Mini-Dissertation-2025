% Options for packages loaded elsewhere
% Options for packages loaded elsewhere
\PassOptionsToPackage{unicode}{hyperref}
\PassOptionsToPackage{hyphens}{url}
\PassOptionsToPackage{dvipsnames,svgnames,x11names}{xcolor}
%
\documentclass[
  a4paper,
  DIV=11,
  numbers=noendperiod,
  twoside]{scrreprt}
\usepackage{xcolor}
\usepackage{amsmath,amssymb}
\setcounter{secnumdepth}{5}
\usepackage{iftex}
\ifPDFTeX
  \usepackage[T1]{fontenc}
  \usepackage[utf8]{inputenc}
  \usepackage{textcomp} % provide euro and other symbols
\else % if luatex or xetex
  \usepackage{unicode-math} % this also loads fontspec
  \defaultfontfeatures{Scale=MatchLowercase}
  \defaultfontfeatures[\rmfamily]{Ligatures=TeX,Scale=1}
\fi
\usepackage{lmodern}
\ifPDFTeX\else
  % xetex/luatex font selection
\fi
% Use upquote if available, for straight quotes in verbatim environments
\IfFileExists{upquote.sty}{\usepackage{upquote}}{}
\IfFileExists{microtype.sty}{% use microtype if available
  \usepackage[]{microtype}
  \UseMicrotypeSet[protrusion]{basicmath} % disable protrusion for tt fonts
}{}
\makeatletter
\@ifundefined{KOMAClassName}{% if non-KOMA class
  \IfFileExists{parskip.sty}{%
    \usepackage{parskip}
  }{% else
    \setlength{\parindent}{0pt}
    \setlength{\parskip}{6pt plus 2pt minus 1pt}}
}{% if KOMA class
  \KOMAoptions{parskip=half}}
\makeatother
% Make \paragraph and \subparagraph free-standing
\makeatletter
\ifx\paragraph\undefined\else
  \let\oldparagraph\paragraph
  \renewcommand{\paragraph}{
    \@ifstar
      \xxxParagraphStar
      \xxxParagraphNoStar
  }
  \newcommand{\xxxParagraphStar}[1]{\oldparagraph*{#1}\mbox{}}
  \newcommand{\xxxParagraphNoStar}[1]{\oldparagraph{#1}\mbox{}}
\fi
\ifx\subparagraph\undefined\else
  \let\oldsubparagraph\subparagraph
  \renewcommand{\subparagraph}{
    \@ifstar
      \xxxSubParagraphStar
      \xxxSubParagraphNoStar
  }
  \newcommand{\xxxSubParagraphStar}[1]{\oldsubparagraph*{#1}\mbox{}}
  \newcommand{\xxxSubParagraphNoStar}[1]{\oldsubparagraph{#1}\mbox{}}
\fi
\makeatother


\usepackage{longtable,booktabs,array}
\usepackage{calc} % for calculating minipage widths
% Correct order of tables after \paragraph or \subparagraph
\usepackage{etoolbox}
\makeatletter
\patchcmd\longtable{\par}{\if@noskipsec\mbox{}\fi\par}{}{}
\makeatother
% Allow footnotes in longtable head/foot
\IfFileExists{footnotehyper.sty}{\usepackage{footnotehyper}}{\usepackage{footnote}}
\makesavenoteenv{longtable}
\usepackage{graphicx}
\makeatletter
\newsavebox\pandoc@box
\newcommand*\pandocbounded[1]{% scales image to fit in text height/width
  \sbox\pandoc@box{#1}%
  \Gscale@div\@tempa{\textheight}{\dimexpr\ht\pandoc@box+\dp\pandoc@box\relax}%
  \Gscale@div\@tempb{\linewidth}{\wd\pandoc@box}%
  \ifdim\@tempb\p@<\@tempa\p@\let\@tempa\@tempb\fi% select the smaller of both
  \ifdim\@tempa\p@<\p@\scalebox{\@tempa}{\usebox\pandoc@box}%
  \else\usebox{\pandoc@box}%
  \fi%
}
% Set default figure placement to htbp
\def\fps@figure{htbp}
\makeatother


% definitions for citeproc citations
\NewDocumentCommand\citeproctext{}{}
\NewDocumentCommand\citeproc{mm}{%
  \begingroup\def\citeproctext{#2}\cite{#1}\endgroup}
\makeatletter
 % allow citations to break across lines
 \let\@cite@ofmt\@firstofone
 % avoid brackets around text for \cite:
 \def\@biblabel#1{}
 \def\@cite#1#2{{#1\if@tempswa , #2\fi}}
\makeatother
\newlength{\cslhangindent}
\setlength{\cslhangindent}{1.5em}
\newlength{\csllabelwidth}
\setlength{\csllabelwidth}{3em}
\newenvironment{CSLReferences}[2] % #1 hanging-indent, #2 entry-spacing
 {\begin{list}{}{%
  \setlength{\itemindent}{0pt}
  \setlength{\leftmargin}{0pt}
  \setlength{\parsep}{0pt}
  % turn on hanging indent if param 1 is 1
  \ifodd #1
   \setlength{\leftmargin}{\cslhangindent}
   \setlength{\itemindent}{-1\cslhangindent}
  \fi
  % set entry spacing
  \setlength{\itemsep}{#2\baselineskip}}}
 {\end{list}}
\usepackage{calc}
\newcommand{\CSLBlock}[1]{\hfill\break\parbox[t]{\linewidth}{\strut\ignorespaces#1\strut}}
\newcommand{\CSLLeftMargin}[1]{\parbox[t]{\csllabelwidth}{\strut#1\strut}}
\newcommand{\CSLRightInline}[1]{\parbox[t]{\linewidth - \csllabelwidth}{\strut#1\strut}}
\newcommand{\CSLIndent}[1]{\hspace{\cslhangindent}#1}



\setlength{\emergencystretch}{3em} % prevent overfull lines

\providecommand{\tightlist}{%
  \setlength{\itemsep}{0pt}\setlength{\parskip}{0pt}}



 


\KOMAoption{captions}{tableheading}
\makeatletter
\@ifpackageloaded{bookmark}{}{\usepackage{bookmark}}
\makeatother
\makeatletter
\@ifpackageloaded{caption}{}{\usepackage{caption}}
\AtBeginDocument{%
\ifdefined\contentsname
  \renewcommand*\contentsname{Table of contents}
\else
  \newcommand\contentsname{Table of contents}
\fi
\ifdefined\listfigurename
  \renewcommand*\listfigurename{List of Figures}
\else
  \newcommand\listfigurename{List of Figures}
\fi
\ifdefined\listtablename
  \renewcommand*\listtablename{List of Tables}
\else
  \newcommand\listtablename{List of Tables}
\fi
\ifdefined\figurename
  \renewcommand*\figurename{Figure}
\else
  \newcommand\figurename{Figure}
\fi
\ifdefined\tablename
  \renewcommand*\tablename{Table}
\else
  \newcommand\tablename{Table}
\fi
}
\@ifpackageloaded{float}{}{\usepackage{float}}
\floatstyle{ruled}
\@ifundefined{c@chapter}{\newfloat{codelisting}{h}{lop}}{\newfloat{codelisting}{h}{lop}[chapter]}
\floatname{codelisting}{Listing}
\newcommand*\listoflistings{\listof{codelisting}{List of Listings}}
\makeatother
\makeatletter
\makeatother
\makeatletter
\@ifpackageloaded{caption}{}{\usepackage{caption}}
\@ifpackageloaded{subcaption}{}{\usepackage{subcaption}}
\makeatother
\usepackage{bookmark}
\IfFileExists{xurl.sty}{\usepackage{xurl}}{} % add URL line breaks if available
\urlstyle{same}
\hypersetup{
  pdftitle={Quantifying Pressing Effectiveness and Its Impact on Formations in Football},
  pdfauthor={Kenneth Ssekimpi},
  pdfkeywords={Football Analytics, Pressing, VPEP, Machine
Learning, Defensive Metrics},
  colorlinks=true,
  linkcolor={blue},
  filecolor={Maroon},
  citecolor={Blue},
  urlcolor={Blue},
  pdfcreator={LaTeX via pandoc}}


\title{Quantifying Pressing Effectiveness and Its Impact on Formations
in Football}
\author{Kenneth Ssekimpi}
\date{}
\begin{document}
\maketitle

\renewcommand*\contentsname{Table of contents}
{
\hypersetup{linkcolor=}
\setcounter{tocdepth}{2}
\tableofcontents
}

\bookmarksetup{startatroot}

\chapter*{Glossary}\label{glossary}
\addcontentsline{toc}{chapter}{Glossary}

\markboth{Glossary}{Glossary}

\section*{Abbreviations}\label{abbreviations}
\addcontentsline{toc}{section}{Abbreviations}

\markright{Abbreviations}

\begin{description}
\item[AI]
Artificial Intelligence
\item[CNN]
Convolutional Neural Network
\item[GAN]
Generative Adversarial Network
\item[CNN]
Convolutional Neural Network
\item[GAN]
Generative Adversarial Network
\item[GRU]
Gated Recurrent Unit
\item[LSTM]
Long Short-Term Memory
\item[MLP]
Multi-Layer Perceptron
\item[RBM]
Restricted Boltzmann Machine
\item[TCN]
Temporal Convolutional Network
\item[VAE]
Variational Autoencoder
\item[WD]
Wasserstein-Disrtance
\item[cVAE]
Conditional Variational Autoencoder
\item[GAIN]
Generative Adversarial Imputation Networks
\item[RMS(L)E]
Root Mean Squared Error
\item[G-means]
G-means
\item[NLL]
Negative Log-Likelihood
\item[MCC]
Matthews Correlation Coefficient
\end{description}

\section*{Parameters}\label{parameters}
\addcontentsline{toc}{section}{Parameters}

\markright{Parameters}

\begin{description}
\item[\(\text{Fe}\)]
Iron (mg/L)
\item[\(\text{Al}\)]
Aluminum (mg/L)
\end{description}

\section*{Nomenclature}\label{nomenclature}
\addcontentsline{toc}{section}{Nomenclature}

\markright{Nomenclature}

\begin{description}
\item[\(J\)]
cost/loss function
\item[\(\nabla\)]
derivative of cost w.r.t. model parameters
\item[\(w\)]
model parameters
\item[\(b\)]
model parameters
\item[\(x\)]
input data
\item[\(y\)]
output data
\item[\(\mu\)]
mean
\item[\(\sigma^2\)]
variance
\end{description}

\section*{Terminology}\label{terminology}
\addcontentsline{toc}{section}{Terminology}

\markright{Terminology}

\begin{description}
\item[gradient]
derivative of cost w.r.t. model parameters
\item[optimizer]
algorithm used to minimize the cost function
\item[batch]
subset of training data used in one iteration
\item[epoch]
one pass through the entire training dataset
\item[overfitting]
model is memorizing the training data and not generalizing well to new
data
\item[underfitting]
model is not able to learn the underlying patterns in the data
\item[regularization]
optimisation by adding a penalty term to the loss function
\item[dropout]
optimise by randomly dropping out neurons during training
\end{description}

\bookmarksetup{startatroot}

\chapter{Introduction}\label{introduction}

\section{Background}\label{background}

Association football, also known as soccer, is a game played by two
teams of 11 players each. Teams compete by advancing a ball into their
opponent's goal, adhering to established rules that govern gameplay,
player conduct, and scoring, with the aim to score more goals than their
rivals (Sumpter, 2016; Memmert \& Raabe, 2018). Although the fundamental
simplicity of football contributes significantly to its global
popularity and appeal, the game simultaneously possesses incredible
complexity characterised by movement patterns, match plans, playing
philosophies, and creativity (Bornn, Cervone \& Fernandez, 2018; Vicente
et al., 2024). These qualities are collectively referred to as football
tactics (Memmert \& Raabe, 2018).

Football tactics involve strategically positioning players on the field
and co-ordinating their movements to maximise the chances of winning
matches. This encompasses both the formation that a team adopts (that
is, the spatial arrangement of players on the pitch) and their overall
style of play (Wilson, 2010). Furthermore, Rein \& Memmert (2016)
describe football tactics as the actions and strategies implemented by a
team and its players during a match to achieve specific goals, primarily
winning the game. These actions are typically adaptations to
dynamically-changing situations in the match and the behaviour of the
opposing team, managing space, time, and individual actions on the
pitch.

Fundamental to understanding these tactical deployments are the distinct
player positions and the common tactical formations they operate within.
Football teams are typically composed of players in four main
categories: Goalkeepers (GK), primarily responsible for preventing
goals; Defenders (DEF), who protect their goal and prevent opposing
attacks (e.g., Centre-Backs, Full-Backs); Midfielders (MID), who link
defence and attack and control central areas (e.g., Defensive
Midfielders, Attacking Midfielders, Wingers); and Forwards/Strikers
(FWD), whose main role is to score goals. These positions are
strategically arranged into tactical formations, which dictate a team's
general shape and approach during different phases of play. A phase of
play in football refers to a segment of a match, typically defined for
analysis or coaching purposes, characterised by the actions and
organisation of one or both teams (Ghezzi \& Sotudeh, 2024). It can be
understood as a sequence of consecutive events or actions that fit
together (Decroos, 2020)

\bookmarksetup{startatroot}

\chapter*{References}\label{references}
\addcontentsline{toc}{chapter}{References}

\markboth{References}{References}

\phantomsection\label{refs}
\begin{CSLReferences}{0}{0}
\bibitem[\citeproctext]{ref-bornn_soccer_2018}
Bornn, L., Cervone, D. \& Fernandez, J. 2018. Soccer analytics:
{Unravelling} the {Complexity} of {``{The} {Beautiful} {Game}''}.
\emph{Significance}. 15(3):26--29.

\bibitem[\citeproctext]{ref-decroos_soccer_2020}
Decroos, T. 2020. Soccer {Analytics} {Meets} {Artificial}
{Intelligence}: {Learning} {Value} and {Style} from {Soccer} {Event}
{Stream} {Data}. PhD thesis. KU Leuven. Available:
\url{https://tomdecroos.github.io/reports/thesis_tomdecroos.pdf}.

\bibitem[\citeproctext]{ref-ghezzi2024}
Ghezzi, E. \& Sotudeh, H. 2024. Hudl StatsBomb conference 2024. Old
Trafford, Manchester, United Kingdom: StatsBomb. 13. Available:
\url{https://statsbomb.com/wp-content/uploads/2024/10/Match-Phases-In-Practice-Ghezzi-and-Sotudeh.pdf}.

\bibitem[\citeproctext]{ref-memmert_data_2018}
Memmert, D. \& Raabe, D. 2018. \emph{Data {Analytics} in {Football}:
{Positional} {Data} {Collection}, {Modelling} and {Analysis}}. 1st ed.
Taylor \& Francis. Available:
\url{https://books.google.co.za/books?id=O9tdDwAAQBAJ}.

\bibitem[\citeproctext]{ref-rein_big_2016}
Rein, R. \& Memmert, D. 2016. Big data and tactical analysis in elite
soccer: Future challenges and opportunities for sports science.
\emph{SpringerPlus}. 5(1):1410. DOI:
\href{https://doi.org/10.1186/s40064-016-3108-2}{10.1186/s40064-016-3108-2}.

\bibitem[\citeproctext]{ref-sumpter_soccermatics_2016}
Sumpter, D. 2016. \emph{Soccermatics: {Mathematical} {Adventures} in the
{Beautiful} {Game}}. 1st ed. Bloomsbury Publishing. Available:
\url{https://books.google.co.za/books?id=CoZVCwAAQBAJ}.

\bibitem[\citeproctext]{ref-vicente_why_2024}
Vicente, L.N., Alleck, T., Giovannelli, T., Mitchell, R. \& Remen, O.
2024. Why is soccer so popular: {Understanding} underdog achievement and
randomness in team ball sports. \emph{arXiv preprint arXiv:2404.06626}.

\bibitem[\citeproctext]{ref-wilson_inverting_2010}
Wilson, J. 2010. \emph{Inverting the {Pyramid}: {The} {History} of
{Football} {Tactics}}. 1st ed. Orion Books Ltd. Available:
\url{https://books.google.co.za/books?id=qYSAhJn-srwC}.

\end{CSLReferences}




\end{document}
