%%%%% CHOOSE YOUR LINE SPACING HERE
% This is the official option.  Use it for your submission copy and library copy:
\setlength{\textbaselineskip}{22pt plus2pt}
% This is closer spacing (about 1.5-spaced) that you might prefer for your personal copies:
%\setlength{\textbaselineskip}{18pt plus2pt minus1pt}

% You can set the spacing here for the roman-numbered pages (acknowledgements, table of contents, etc.)
\setlength{\frontmatterbaselineskip}{17pt plus1pt minus1pt}

% Leave this line alone; it gets things started for the real document.
\setlength{\baselineskip}{\textbaselineskip}

%%%%% CHOOSE YOUR SECTION NUMBERING DEPTH HERE
% You have two choices.  First, how far down are sections numbered?  (Below that, they're named but
% don't get numbers.)  Second, what level of section appears in the table of contents?  These don't have
% to match: you can have numbered sections that don't show up in the ToC, or unnumbered sections that
% do.  Throughout, 0 = chapter; 1 = section; 2 = subsection; 3 = subsubsection, 4 = paragraph...

% The level that gets a number:
\setcounter{secnumdepth}{2}
% The level that shows up in the ToC:
\setcounter{tocdepth}{2}


%%%%% ABSTRACT SEPARATE
% This is used to create the separate, one-page abstract that you are required to hand into the Exam
% Schools.  You can comment it out to generate a PDF for printing or whatnot.
\begin{abstractseparate}
	\input{frontmatter/03-abstract.txt} % Create an abstract.tex file in the 'text' folder for your abstract.
\end{abstractseparate}

\newpage
\thispagestyle{empty}
\null
\newpage
% JEM: Pages are roman numbered from here, though page numbers are invisible until ToC.  This is in
% keeping with most typesetting conventions.
\pagenumbering{roman}

\begin{romanpages}

% Title page is created here
\maketitle

%%%%% DEDICATION -- If you'd like one, un-comment the following.

\begin{dedication}
    \input{frontmatter/01-dedication.txt}
\end{dedication}

%%%%% Declaration -- Nothing to do here except comment out if you don't want it.

\begin{declaration}
\addcontentsline{toc}{chapter}{Declaration}
	\pagenumbering{gobble}

\vspace*{\fill}

\noindent \textbf{\GetTitle}

\noindent Copyright \textcopyright~\the\year{} - \GetAuthor, \GetFaculty.

\vspace{.575em}

\noindent I know the meaning of plagiarism and declare that all of the work in this dissertation, save for which is properly acknowledged, is my own. I have not previously submitted this work, in whole or in part, for any other degree or qualification at this or any other institution.---
\textit{University of Cape Town}---.

\vspace{1.395em}

\noindent\psvectorian[scale=.25,opacity=.80]{2}

\vspace{.935em}

\noindent Preparation of this work was facilitated by the use if the \textit{IPLeiria-Thesis} template.

\vspace*{\fill}
\MediaOptionLogic
    \vspace{0.5em}
    %\begin{flushright}
      %\includegraphics[width=4 cm]{frontmatter/figs/signature.jpg}
    %\end{flushright}
    \vspace{0.5em}
    \begin{flushright}
      {\large \textit{$author$}}
    \end{flushright}
    \vspace{0.5em}
    \noindent Student Number: $studentnumber$
    \vspace{0.5em}
    \noindent Word Count: $wordcount$
    \vfill
    \noindent This dissertation is submitted for the degree of $degree$ in the $department$, $faculty$, at the $university$.
    \vspace{2cm}
    \noindent This minor dissertation was conducted under the supervision of $supervisor$ ($supervisortitle$).
    \vspace{2cm}
\end{declaration}

%%%%% ABSTRACT -- Nothing to do here except comment out if you don't want it.

\begin{abstract}
\addcontentsline{toc}{chapter}{Abstract}
	\input{frontmatter/03-abstract.txt}
\end{abstract}

%%%%% ACKNOWLEDGEMENTS -- Nothing to do here except comment out if you don't want it.

\begin{acknowledgements}
\addcontentsline{toc}{chapter}{Acknowledgements}
  \noindent I would like to express my sincere gratitude to my supervisor, $supervisor$, for his invaluable guidance, support, and patience throughout this research. His expertise and insights have been instrumental in the completion of this dissertation.
 	\input{frontmatter/04-acknowledgements.txt}
\end{acknowledgements}

% This aligns the bottom of the text of each page.  It generally makes things look better.
\flushbottom
