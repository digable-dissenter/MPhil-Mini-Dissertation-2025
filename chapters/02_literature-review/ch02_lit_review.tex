\chapter{Literature Review}
\section{Introduction}
Pressing in football has evolved into one of the defining features of modern tactics, linking defensive resilience with offensive opportunity. 

\section{The Evolution and Typology of Pressing Strategies}

\subsection{Historical Development of Pressing}



\subsection{Typology of Pressing Strategies}



\subsection{Tactical Concepts Within Pressing}



\subsection{Tactical Trade-Offs}

\section{Measuring and Analysing Pressing Effectiveness}

\subsection{Traditional Metrics: Intent vs. Effectiveness}


\subsection{Granular Metrics from Event Data}


\subsection{Positional and Spatio-Temporal Analysis}



\subsection{Advanced Modelling Approaches}



\subsection{Visual and Qualitative Analysis}




\subsection{Pressing in Context}




\subsection{Linking Pressing to Threat Models}




\subsection{Summary}


\section{The Role of Data and Positional Analysis in Football}

The value of pressing can be quantified using data-driven metrics that assess a player's ability to intercept passes and disrupt the opponent's play. These metrics help in evaluating defensive performances and identifying key players in pressing roles.

Advanced models can rank players based on their pressing effectiveness, which can be useful for scouting and tactical planning.

\subsection{The Impact of Positional Data}


\subsection[{The Data Revolution]{The Data Revolution: Key Elements \& Technologies}




\subsection{Challenges and Considerations}


\section{The Influence of Team Formations on Pressing}


\subsection{Formation Influence on Pressing Dynamics}


\subsection{Formation Disruption}


\subsection{Need for Integrated Analysis}


\section{Physical and Tactical Correlations}


\subsection{Physical Demands of Pressing}


\subsection{Tactical Efficiency and Physical Load}


\subsection{Critique}





\section{Synthesis and Identification of Research Gap}

While pressing is a highly valued aspect of football, it is not without its challenges. The physical demands of pressing can lead to fatigue and increase the risk of injuries, particularly if not managed properly. Additionally, pressing requires a high level of discipline and understanding among players, as mistimed or uncoordinated pressing can leave a team vulnerable to counter-attacks. Therefore, while pressing is a powerful tool, it must be implemented with careful consideration of the team's capabilities and the specific context of the match.



\section{Research Opportunity and Contribution}