\chapter{Literature Review}
\section{Introduction}
Pressing in football has evolved into one of the defining features of modern tactics, linking defensive resilience with offensive opportunity. 

\section{The Evolution and Typology of Pressing Strategies}

\subsection{Historical Development of Pressing}



\subsection{Typology of Pressing Strategies}



\subsection{Tactical Concepts Within Pressing}



\subsection{Tactical Trade-Offs}

\section{Measuring and Analysing Pressing Effectiveness}

\subsection{Traditional Metrics: Intent vs. Effectiveness}


\subsection{Granular Metrics from Event Data}


\subsection{Positional and Spatio-Temporal Analysis}



\subsection{Advanced Modelling Approaches}



\subsection{Visual and Qualitative Analysis}




\subsection{Pressing in Context}




\subsection{Linking Pressing to Threat Models}




\subsection{Summary}


\section{The Role of Data and Positional Analysis in Football}

\subsection{The Impact of Positional Data}


\subsection{The Data Revolution: Key Elements & Technologies}


\subsection{Challenges and Considerations}


\section{The Influence of Team Formations on Pressing}


\subsection{Formation Influence on Pressing Dynamics}


\subsection{Formation Disruption}


\subsection{Need for Integrated Analysis}


\section{Physical and Tactical Correlations}


\subsection{Physical Demands of Pressing}


\subsection{Tactical Efficiency and Physical Load}


\subsection{Critique}



\section{Synthesis and Identification of Research Gap}





\section{Research Opportunity and Contribution}